\chapter*{Abstract (English)}


Single-cell analysis provides a new approach to inspect biological processes at a single-cell resolution. Recently, new sequencing protocols have been developed to simultaneously profile the transcriptome, epigenome, and proteome features in single cells. With these technologies, researchers can interpret biological phenomena such as gene regulation and transcription factor binding events more effectively. The analysis of these simultaneous profiling data, namely, single-cell multimodal analysis, is becoming increasingly popular.

However, a major challenge in single-cell multimodal analysis is that the feature spaces of distinct modalities are extremely different. The differences in feature sizes, sparsity, and distributions pose a significant challenge in utilizing the features to obtain a shared, uniform latent space across all modalities. Although several methods have been proposed to address this issue, these methods either require too much prior knowledge to set parameters or lack scalability. Moreover, many methods only work for specific modality types, or it's impossible to interpret the inferred latent components.

Another critical issue when dealing with single-cell multimodal data is to infer trajectories to capture cell lineage development. Many methods have been developed to infer trajectories for single-cell data, but few are designed for single-cell multimodal analysis, which can offer more information for inference. Moreover, many trajectory analysis methods only utilize lower dimensions to infer trajectories, leaving high-dimensional information unused. Furthermore, when dealing with complex datasets with too many lineages, many methods are not capable of inferring appropriate trajectories. Additionally, most methods only use the single nodes, lacking the utilization of signals on edges.

In this thesis, we propose MOJITOO and PHLOWER, accounting respectively for single-cell multimodal integration and trajectory inference. MOJITOO is developed to address the aforementioned issues using Canonical Correlation Analysis (CCA). MOJITOO can effectively and efficiently integrate any types of modality combinations. Moreover, MOJITOO requires no prior knowledge to set up parameters, capturing the most shared latent space. We performed comprehensive benchmarking evaluations to check space preservation, clustering, and distance accuracy, finding that MOJITOO performs the best overall. Furthermore, we demonstrate that the latent space obtained by MOJITOO can capture major cell types or cell type differentiation information. Overall, these results demonstrate that MOJITOO is a powerful computational approach in biological studies for single-cell multimodal integration analysis.

PHLOWER is designed to address the trajectory inference issues that current methods lack. It is designed for trajectory inference for single-cell multimodal data using Hodge Laplacian (HL) decomposition. We show that PHLOWER scores first place in four metrics for the comprehensive trajectory inference evaluation framework using data with different complexities. Moreover, we explore the power of multimodal analysis using PHLOWER on large-scale single-cell multimodal data with transcriptome and epigenome from single-cell kidney organoid data. We are able to detect transcription factors that regulate lineage gene expression. Our analyses shed novel light on mechanisms of kidney lineage development. Altogether, these results demonstrate that PHLOWER is a powerful computational approach in biological studies for single-cell multimodal trajectory inference analysis.

