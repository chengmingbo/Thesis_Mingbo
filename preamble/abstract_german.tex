\chapter*{Abstrakt (Deutsch)}

Die Einzelzellanalyse bietet einen neuen Ansatz zur Untersuchung biologischer Prozesse auf Einzelzellebene. In letzter Zeit wurden neue Sequenzierprotokolle entwickelt, um gleichzeitig das Transkriptom, das Epigenom und das Proteom in Einzelzellen zu profilieren. Mit diesen Technologien können Forscher biologische Phänomene wie die Regulation von Genen und Ereignisse der Transkriptionsfaktorbindung effektiver interpretieren. Die Analyse dieser gleichzeitigen Profilierungsdaten, auch als Einzelzell-Multimodalanalyse bekannt, erfreut sich zunehmender Beliebtheit.

Eine wesentliche Herausforderung bei der Einzelzell-Multimodalanalyse besteht jedoch darin, dass die Merkmalsräume verschiedener Modalitäten extrem unterschiedlich sind. Die Unterschiede in der Merkmalsgröße, der Sparsamkeit und den Verteilungen stellen eine erhebliche Herausforderung dar, um die Merkmale zu nutzen und einen gemeinsamen, einheitlichen latenten Raum über alle Modalitäten hinweg zu erhalten. Obwohl mehrere Methoden vorgeschlagen wurden, um dieses Problem anzugehen, erfordern diese Methoden entweder zu viel Vorwissen, um Parameter einzustellen, oder sie mangelt an Skalierbarkeit. Darüber hinaus funktionieren viele Methoden nur für bestimmte Modalitätstypen, oder es ist unmöglich, die abgeleiteten latenten Komponenten zu interpretieren.

Ein weiteres wichtiges Problem beim Umgang mit Einzelzell-Multimodaldaten besteht darin, Trajektorien zu inferieren, um die Entwicklung von Zelllinien zu erfassen. Viele Methoden wurden entwickelt, um Trajektorien für Einzelzellerdaten zu inferieren, aber nur wenige sind für die Einzelzell-Multimodalanalyse konzipiert, die mehr Informationen für die Inferenz bieten kann. Darüber hinaus nutzen viele Trajektorienanalysen nur niedrigere Dimensionen, um Trajektorien zu inferieren, wobei hochdimensionale Informationen ungenutzt bleiben. Wenn es um komplexe Datensätze mit zu vielen Linien geht, sind viele Methoden nicht in der Lage, geeignete Trajektorien zu inferieren. Darüber hinaus verwenden die meisten Methoden nur einzelne Knoten und nutzen nicht die Signale auf den Kanten.

In dieser Arbeit schlagen wir MOJITOO und PHLOWER vor, die jeweils für die Einzelzell-Multimodalintegration und die Trajektorieninferenz entwickelt wurden. MOJITOO wurde entwickelt, um die zuvor genannten Probleme mit Hilfe der kanonischen Korrelationsanalyse (CCA) zu lösen. MOJITOO kann effektiv und effizient jede Art von Modalitätskombination integrieren. Darüber hinaus benötigt MOJITOO kein Vorwissen, um Parameter einzustellen, sondern erfasst den gemeinsamsten latenten Raum. Wir führten umfassende Benchmark-Evaluierungen durch, um Raumkonservierung, Clustering und Distanzgenauigkeit zu überprüfen, und stellten fest, dass MOJITOO insgesamt am besten abschneidet. Darüber hinaus zeigen wir, dass der von MOJITOO erhaltene latente Raum wesentliche Zelltypen oder Informationen zur Zelltyp-Differenzierung erfassen kann. Insgesamt zeigen diese Ergebnisse, dass MOJITOO ein leistungsfähiger rechnerischer Ansatz in biologischen Studien für die Einzelzell-Multimodalintegrationsanalyse ist.

PHLOWER wurde entwickelt, um die Trajektorieninferenzprobleme zu lösen, die aktuellen Methoden fehlen. Es ist für die Trajektorieninferenz für Einzelzell-Multimodaldaten unter Verwendung der Hodge-Laplacian(HL)-Zerlegung konzipiert. Wir zeigen, dass PHLOWER in vier Metriken für das umfassende Trajektorieninferenz-Bewertungsframework unter Verwendung von Daten mit unterschiedlichen Komplexitäten den ersten Platz erreicht. Darüber hinaus erforschen wir die Leistung der Multimodalanalyse unter Verwendung von PHLOWER an großen Einzelzell-Multimodaldaten mit Transkriptom und Epigenom von Einzelzell-Nierenorganoiddaten. Wir können Transkriptionsfaktoren nachweisen, die die Expression von Genen in einer Linie regulieren. Unsere Analysen werfen neues Licht auf die Mechanismen der Entwicklung der Nierenlinie. Insgesamt zeigen diese Ergebnisse, dass PHLOWER ein leistungsfähiger rechnerischer Ansatz in biologischen Studien für die Einzelzell-Mult